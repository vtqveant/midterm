\documentclass[10pt]{article} 

\usepackage[T2A]{fontenc}
\usepackage[utf8]{inputenc}
\usepackage[english,russian]{babel}
\usepackage{amssymb,amsfonts,amsmath,amsthm,amscd,mathtext}

\usepackage[a4paper]{geometry} 
\geometry{verbose,tmargin=2cm,bmargin=2cm,lmargin=2cm,rmargin=2cm}

\sloppy
\clubpenalty=10000
\widowpenalty=10000

%%% оформление заголовков глав, секций и подсекций

\usepackage{titlesec}

\makeatletter
  \renewcommand{\section}{\@startsection{section}{1}{5.5ex}{-3.5ex plus -1ex minus -.2ex}{3.2ex plus.2ex}{\raggedright\hyphenpenalty=10000\normalfont\bfseries}}
\makeatother

\usepackage{indentfirst}      % красная строка
\setlength{\parindent}{5.5ex} % 5 символов

%%% math env

\newtheoremstyle{example-style}% name
{10pt}  % space above 
{10pt}  % space below 
{}     % body font
{\parindent}  % indent 
{\bfseries}  % theorem head font
{.}  % punctuation after theorem head
{.5em}  % space after theorem head 
{}  % theorem head spec (can be left empty, meaning 'normal')

\theoremstyle{example-style}
\newtheorem{example}{Задание}

\usepackage[nosolutionfiles]{answers}
\Newassociation{sol}{Solution}{ans}


\begin{document} 

\title{Анализ текстов на естественных языках\ \\{\large Контрольная работа}}
\author{}
\date{}

\maketitle

\Opensolutionfile{ans}[ans1]


\begin{example}
Дан язык $L = \{aa, bb, abab, baba, aaaa, bbbb\}$ над алфавитом $\Sigma = \{a, b\}$. Вычислить оценки максимального правдоподобия для параметров биграммной языковой модели, с их помощью вычислить вероятность строки $abba$.
\end{example}

\begin{example}
Звуки в естественных языках можно разделить на гласные, согласные и непонятные. В зависимости от позиции в слоге непонятные звуки могут играть роль как гласных (например, нести ударение), так и согласных (например, закрывать слог и влиять на длительность или оттенок предшествующего гласного). Дан алфавит $\Sigma = \{a, b, j\}$, набор помет $T = \{C, V\}$ и обучающая выборка $(\mathcal{X}, \mathcal{Y}) = \{(a, V), (ba, CV), (baj, CVC), (ajb, VVC), (jab, CVC), (bja, CVV) \}$. Требуется определить модель слоговой структуры как биграммную скрытую марковскую модель и с её помощью найти слоговую структуру (наиболее вероятную аннотацию) строки $jaj$. Параметры модели задать на основе оценок максимального правдоподобия.
\end{example}

\begin{example}
Требуется построить систему автоматической расстановки переносов для русского языка на основе скрытой марковской модели. Выпишите спецификацию модели.
\end{example}

\begin{example}
Дана вероятностная контекстно-свободная грамматика $\langle N, \Sigma, R, S, q \rangle$, где $N = \{ A, B, C, S \}$, $\Sigma = \{a, b\}$,
$R = \{ S \to CB \, | \, BC, \; C \to BA \, | \, AB, \; A \to a, B \to b \}$, 
$q(S \to CB) = 0.7, q(S \to BC) = 0.3, q(C \to BA) = 0.4, q(C \to AB) = 0.6, q(A \to a) = 1.0, q(B \to b) = 1.0$.
Найти наилучший (наиболее вероятный) разбор для строки $abb$.
\end{example}

\Closesolutionfile{ans}


\end{document}